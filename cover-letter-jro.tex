\documentclass[letterpaper, 10pt]{letter}
\usepackage{anysize}
\marginsize{1in}{1in}{0.5in}{0.5in}
\usepackage{setspace}
\usepackage{url}
\usepackage{verbatim}
\usepackage{parskip}
\usepackage{graphicx}
\usepackage{hyperref}
\hypersetup{pdfborder={0 0 0}, colorlinks=true, urlcolor=blue}

\newcommand{\ignore}[1]{}
\newcommand{\super}[1]{\ensuremath{^{\textrm{#1}}}}
\newcommand{\sub}[1]{\ensuremath{_{\textrm{#1}}}}
\newcommand{\dC}{\ensuremath{^\circ{\textrm{C}}}}

\makeatletter
\let\@texttop\relax
\makeatother

\newenvironment{myEnumerate}{
  \begin{enumerate}
    \setlength{\itemsep}{1pt}
    \setlength{\parskip}{0pt}
    \setlength{\parsep}{0pt}}
  {\end{enumerate}}

\newenvironment{myItemize}{
  \begin{itemize}
    \setlength{\itemsep}{1pt}
    \setlength{\parskip}{0pt}
    \setlength{\parsep}{0pt}}
  {\end{itemize}}

%%%%%%%%%%%%%%%%%%%%%%%%%%%%%%%%%%%%%%%%%%%%%%%%%%%%%%%%%%%%%%

% \date{October 14, 2012}
\signature{\includegraphics[scale=0.5]{signature.jpg} \\
                 \medskip
                 Jamie Oaks \\
                 Postdoctoral Fellow \\ 
                 Department of Biology \\
                 University of Washington \\ 
                 Box 351800 \\
                 Seattle, Washington 98195 \\ 
                 \href{mailto:joaks1@uw.edu}{\texttt{joaks1@uw.edu}} \\ 
                 802-280-5843}
%\address{Biodiversity Institute \\ University of Kansas \\ Dyche Hall, 1345
                 %Jayhawk Blvd \\ Lawrence, KS 66045}
\begin{document}
\begin{letter}{Dr.\ Suzy C.\ P.\ Renn \\
                     ATTN: Computational Biologist Search Committee \\
                     Biology Department \\
                     Reed College \\
                     3203 SE Woodstock Boulevard \\
                     Portland, OR 97202}
\opening{Dear Dr.\ Renn and Members of the Search Committee:}
% \raggedright
Please consider my application for an Assistant Professorship in Computational
Biology.
I am a NSF Postdoctoral Research Fellow working with Drs.\ Adam Leach\'{e} and
Vladimir Minin in the Departments of Biology and Statistics, respectively, at
the University of Washington.
I am excited about this opportunity in your department because of the unique
intellectual setting it provides.
Your strong commitment to both research and research-based undergraduate
education is a perfect fit for my career aspirations.
Furthermore, my wife attended Lewis and Clark College, and Portland is our top
choice to live and raise our daughter.  Put simply, this is my dream job.

I am a broadly trained biologist who studies the evolutionary genetics of
natural populations.
In my research program, I use genomic variation within and among species to
investigate the processes that generate biodiversity and promote and constrain
adaptation.
My research is quantitative and integrative, incorporating concepts and
techniques from phylogenetics, population genetics, genomics, bioinformatics,
statistics, and computer science.

My qualifications for this faculty position include successful acquisition
of funding for my research program.
During my graduate and postdoctoral career, I have secured more than
\$400,000 in fellowships and grants from such funding sources as the 
National Science Foundation,
National Research Council,
National Geographic Society,
Society of Systematic Biologists, and
Sigma Xi Scientific Research Society.
With this support, I have
(1) conducted extensive international fieldwork that resulted in
the deposition of valuable specimens and genetic materials in multiple
museums around the world,
(2) published 13 peer-reviewed papers, and
(3) developed a number of computational and bioinformatics tools.
% To date, my research has resulted in 13 peer-reviewed publications,
% including five first-authored publications.
% \emph{Evolution},
% \emph{BMC Evolutionary Biology},
% \href{http://onlinelibrary.wiley.com/doi/10.1111/j.1558-5646.2011.01373.x/abstract}{\it
% Evolution}, one of which was featured on the journal's cover and in
% \href{http://www.nature.com/nature/journal/v474/n7353/full/474545a.html}{\it
% Nature}. 

I have a strong background in computational biology and bioinformatics.
Early in my graduate career, I realized that too often scientific questions and
projects are framed around available methods and software.
As a result, I have sought to mitigate this constraint by acquiring the
necessary computational and mathematical skills to be able to develop and
implement the most appropriate statistical methods for addressing my biological
questions of interest.
I am comfortable programming in several different languages, and have spent the
last five years as a developer of several
\href{http://www.phyletica.com/?page_id=249}{open-source software packages}.
% \href{http://phylo.bio.ku.edu/software/sate/sate.html}{{SAT}\'{e}}.
% Furthermore, I write and maintain several other open-source software
% \href{https://github.com/joaks1?tab=repositories}{packages and tools}.

I am strongly committed to teaching and mentoring undergraduate
students.
Through my experiences as a tutor, teaching assistant, workshop instructor
and lecturer, I have developed a strong passion for teaching science.
% I recently received an Award for Excellence in Teaching from the
% University of Kansas Division of Biological Sciences for teaching the
% discussion sections of an undergraduate genetics course.
As a graduate student, I received an Award for Excellence in Teaching from the
University of Kansas (KU) Division of Biological Sciences for teaching
Genetics, and I mentored local high school students, KU undergraduates, and
international undergraduate and graduate students in the methods of field-based
biodiversity research.
As part of my postdoctoral fellowship, I am currently developing and
implementing inquiry-based teaching activities, measuring student learning to
assess their effectiveness, and leading an active-learning classroom in
\href{http://courses.biology.washington.edu/biol180/}{Introductory Biology} at
the University of Washington under the mentorship of Dr.\ Scott Freeman.
Taking an evidence-based, high-engagement approach to the classroom has
prepared me for teaching in a liberal arts environment.


% I am also qualified for the curatorial aspect of the position.
% For two years, I was the Curatorial Assistant for the Herpetology
% collection at the University of Kansas, one of the largest collections of
% amphibians and reptiles in the United States.
% In the absence of a Collection Manager, I was responsible for all aspects
% of managing the collection, including accessions, cataloging, database
% management, loans, import/export permits, and general collection and wet
% lab upkeep.
% I made several improvements to the collection database that greatly
% increased the efficiency and accuracy of digitizing new data and
% processing of loans.

% Under my mentorship, Liz Lusher, a KU undergraduate, secured an Undergraduate
% Research Award and successfully completed a phylogeographic study of ringneck
% snakes across the Great Plains.


My research program is geographically focused in Central and Southeast Asia,
creating many international research opportunities for students.
Furthermore, I will leverage my international collaborations to recruit
talented young students to join my lab from countries such as Malaysia,
Thailand, and Mongolia.
Being a first-generation college graduate, I understand what a tremendous
opportunity a world-class liberal arts education can provide for
underprivileged students.
I am also interested in developing projects locally in the Pacific Northwest;
the diversity of habitats across this region provide an excellent opportunity
for establishing a comparative research program investigating the
diversification of the region's rich biota.
% Such work would create many local projects for undergraduate students to
% complement international opportunities.
% I would love the opportunity to foster and grow the herpetology collection
% in your department through active, collections-based research both locally
% and internationally.


As a faculty member, I will strive to be a passionate teacher, supportive
advisor, well-funded, productive researcher, and collaborative colleague.
My commitment to teaching and mentoring, as well as my diverse and integrative
research interests would be an asset to the Biology Department at Reed College.
%After reviewing my application materials, I hope you will strongly consider me
%for the position.

I have requested letters of support from Drs.\
Rafe Brown
(\href{mailto:rafe@ku.edu}{rafe@ku.edu}),
Mark Holder
(\href{mailto:mtholder@ku.edu}{\tt mtholder@ku.edu}),
and
Adam Leach\'{e}
(\href{mailto:leache@uw.edu}{\tt leache@uw.edu}).
% and
% Cameron Siler
% (\href{mailto:camsiler@ou.edu}{\tt camsiler@ou.edu}).
% transcripts from the University of Wisconsin Oshkosh, Louisiana State
% University, and the University of Kansas.
Upon request, I am happy to provide copies of my manuscripts that are currently
in review or in press.
Thank you for considering my application, and I look forward to speaking with
you about this position.
% Please do not hesitate to contact me if you have any questions.

\addtolength{\medskipamount}{-5pt}
\closing{Sincerely,}
\end{letter}
\end{document}

